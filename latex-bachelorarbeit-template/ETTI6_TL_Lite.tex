% 1) pdflatex main
% 2) makeindex.exe %.idx -s etti4.ist
% 3) biber main
% 4) pdflatex main x 2
%%%%%%%%%%%%%%%%%%%%%%%%%%%%%%%%%%%%%%%%%%%%%%%%%%%%%%%%%%%%%%%%%%%%%%%%%%%%%%
\usepackage{lipsum} % generate random text

%-----------------------------------------------------------------------------
% Deutsche Anpassungen
%-----------------------------------------------------------------------------
\usepackage[ngerman]{babel}      % deutsch content->Inhaltsverzeichnis ... 
\RequirePackage[babel]{microtype}
\usepackage[utf8]{inputenc}
%\usepackage[latin1]{inputenc}
\usepackage{inputenc}            % Umlaute
\usepackage{amssymb}             % Pfeile in math 
\usepackage{trfsigns}            % laplace ... Knoten
\RequirePackage{etex}
\usepackage{prettyref}
\usepackage{titleref}


% Font festlegen!
\RequirePackage{courier}								% Font: Schriftart Courier | Setzt Courier als True-Type-Schriftart (ttfamily)
% Change to Helvetica (close to Arial)
\RequirePackage[scaled]{helvet}
\RequirePackage[T1]{fontenc}

\renewcommand\familydefault{\sfdefault}

\usepackage{csquotes}
\usepackage[toc, acronym, nopostdot]{glossaries}
\usepackage{inconsolata}

\RequirePackage[absolute]{textpos}							    % Layout: Absolute Ausrichtung von Text
\RequirePackage[bottom]{footmisc}								% Fußnoten: Positionierung und Verhalten der Fußnoten ändern

\addto\captionsngerman{
  \renewcommand{\figurename}{Abbildung}
  \renewcommand{\tablename}{Tabelle}
  \renewcommand{\lstlistingname}{Quellcode}
}

%-----------------------------------------------------------------------------
% Grafik und Farben
%-----------------------------------------------------------------------------
%\usepackage{showframe} % <--------------------------------------------------- Ausblenden nicht vergessen! 
\usepackage{graphicx}            % Laden von Grafiken
\graphicspath{{./images/}}       % Pfad zu den Bildern
\usepackage{tikz}                % Grafik mit tex erzeugen
\usepackage{multirow}
% \documentclass[xcolor=table]{beamer}


% definierte Farben
% black, blue, brown, cyan, darkgray, gray, green, lightgray, lime, magenta,
% olive, orange, pink, purple, red, teal, violet, white, yellow
% Apricot 	  	  	  	Aquamarine          Bittersweet 	  	  	BlueGreen 
% BlueViolet   	  	    BrickRed            BurntOrange             CadetBlue
% CarnationPink         Cerulean 	  	  	CornflowerBlue          Dandelion
% DarkOrchid 	  	  	Emerald             ForestGreen 	  	  	Fuchsia
% Goldenrod 	  	  	GreenYellow
% JungleGreen 	  	  	Lavender            LimeGreen 	  	  	  	
% Mahogany 	  	  	  	Maroon              Melon 	  	  	  	    MidnightBlue
% Mulberry 	  	  	  	NavyBlue            OliveGreen 	  	  	  	
% OrangeRed 	  	  	Orchid              Peach 	  	  	  	    Periwinkle
% PineGreen 	  	   	Plum                ProcessBlue 	  	  	
% RawSienna 	  	   	RedOrange 	  	  	RedViolet
% Rhodamine 	  	  	RoyalBlue           RoyalPurple 	  	  	RubineRed
% Salmon 	  	  	  	SeaGreen            Sepia 	  	  	  	    SkyBlue
% SpringGreen 	  	  	Tan                 TealBlue 	  	  	  	Thistle
% Turquoise 	  	  	VioletRed 	  	  	  	
% WildStrawberry 	  	YellowGreen 	  	YellowOrange

% RGB define new colors
\definecolor{dkgreen}{rgb}{0,0.6,0}
\definecolor{mauve}{rgb}{0.58,0,0.82}
\definecolor{darkblue}{rgb}{0.1,0,1.0}
\definecolor{darkgreen}{rgb}{0.1,0.8,0.2}
\definecolor{lightblue}{rgb}{0.8,0.85,1}
\definecolor{blue2}{RGB}{0,170,255}
\definecolor{pale turquoise}{rgb}{0.686,0.933,0.933}
\definecolor{light-gray}{gray}{0.95}
\definecolor{ettiOrange}{HTML}{ED6D00}

% Programmcode Farben
\definecolor{mGreen}{rgb}{0,0.6,0}
\definecolor{mGray}{rgb}{0.5,0.5,0.5}
\definecolor{mPurple}{rgb}{0.58,0,0.82}
\definecolor{backgroundColour}{RGB}{240,248,255}

% reuse and mix colors
\colorlet{vlightgray}{lightgray!20}

% mapping - nur die folgenden Farben verwenden
\colorlet{colBackground}{vlightgray}
\colorlet{colPrimary}{blue2}
\colorlet{colSuccess}{green}
\colorlet{colDanger}{red}
\colorlet{colWarning}{magenta!80}
\colorlet{colGrid}{lightgray!40}

% defaults for listings
\colorlet{colKeyword}{magenta}
\colorlet{colComment}{dkgreen}
\colorlet{colString}{mauve}

% defaults for Inhaltsverzeichnis und chapterthumb
\colorlet{colScriptContent}{black}
\colorlet{colScriptCT}{lightblue}



%-----------------------------------------------------------------------------
% Layout
%-----------------------------------------------------------------------------
\parindent0cm
\usepackage{enumitem} % kleinere Abstände

\usepackage{geometry}

			
%\geometry{includehead,includefoot,inner=2.5cm,outer=1.5cm,top=1.5cm,bottom=1.0cm}
\geometry{includehead,includefoot, inner=2.25cm,outer=2.25cm,top=1.5cm,bottom=1.5cm}
\setlength{\footheight}{20.4pt}
\setlength{\headheight}{20.4pt}

\usepackage{siunitx}                 %Mathe Komma einstellen! 
\sisetup{
    detect-family = true,
    %detect-inline-family = math,
    output-decimal-marker = {,}
}


\usepackage{needspace} % Seitenumbruch falls notwendiger Platz nicht mehr vorhanden


\usepackage{rotating}

\usepackage{scrlayer-scrpage}
\pagestyle{scrheadings}
\RequirePackage{xspace}     % Intelligentes Leerzeichen

%-----------------------------------------------------------------------------
% Kopf- und Fußzeile manipulieren
%-----------------------------------------------------------------------------

\renewcommand{\headfont}{\normalfont\sffamily\itshape}          % Kolumnentitel serifenlos
\renewcommand{\pnumfont}{\normalfont\rmfamily}         % Seitennummern typewriter


%-----------------------------------------------------------------------------
% Tabellen
%-----------------------------------------------------------------------------
\usepackage{colortbl}  % Farbige Tabellen
%\usepackage{longtable} % Tabellen die über mehr als eine Seite gehen
\usepackage{booktabs} 
\usepackage{tabularx}
\RequirePackage{tabu}	% Listen:   Erweiterte Listenformatierung tabu Umgebung
\setlength\arrayrulewidth{0.5pt}
\usepackage{makecell}       %Zeilenumbrüche in Tabellen einfügen


% ZUM EINFÜGEN VON BILDERN in TABELLEN
% neuer Befehl: \includegraphicstotab[..]{..}
% Verwendung analog wie \includegraphics
\newlength{\myx} % Variable zum Speichern der Bildbreite
\newlength{\myy} % Variable zum Speichern der Bildhöhe

\newcommand\includegraphicstotab[2][\relax]{%
	% Abspeichern der Bildabmessungen
	\settowidth{\myx}{\includegraphics[{#1}]{#2}}%
	\settoheight{\myy}{\includegraphics[{#1}]{#2}}%
	% das eigentliche Einfügen
	\parbox[c][1.1\myy][c]{\myx}{%
		\includegraphics[{#1}]{#2}}%
}% Ende neuer Befehl


%-----------------------------------------------------------------------------
% Literaturverzeichnis
%-----------------------------------------------------------------------------
\usepackage[
backend=biber,
style=numeric, % style=alphabetic authoryear numeric,
citestyle=numeric,sorting=none,sortcites=true,autopunct=true,
hyperref=true,abbreviate=false,backref=true,block=ragged,
]{biblatex}
\addbibresource{./Literatur.bib} % Literaturdatenbank
\defbibheading{bibempty}{}


%-----------------------------------------------------------------------------
% Index
%-----------------------------------------------------------------------------
\usepackage{makeidx} % Für das Erstellen von einem Index
\makeindex


%-----------------------------------------------------------------------------
% Datum und Zeit
%-----------------------------------------------------------------------------
\usepackage{scrtime}
%-----------------------------------------------------------------------------
% Text
%-----------------------------------------------------------------------------
%\usepackage{microtype} % besserer Wörterabstand und weniger Textbreitenfehler

\RequirePackage[T1]{fontenc}
\RequirePackage{lmodern}
\usepackage{textcomp}
%\usepackage[scale=0.98,ttdefault]{AnonymousPro}
\usepackage{times}

\usepackage{exscale}   % schöne Formeln in PDFs

%-----------------------------------------------------------------------------
% Verlinkungen im Text
%-----------------------------------------------------------------------------
\usepackage[pdftex,bookmarks=true]{hyperref}
\hypersetup{
    pdfborder={0 0 0}, %Es wird kein Rahmen um irgendeine Verlinkung angezeigt, Verlinkungen bleiben aber bestehen
    bookmarksopen=true,
    bookmarksnumbered=true,
    pdftoolbar=true,
    pdfmenubar=true,
    pdffitwindow=true,
    pdfstartview={FitBH},
    pdftitle={Bachelorarbeit},
    pdfauthor={Lukas Diedenhofen},
    pdfsubject={Evaluation von Frameworks zur plattformunabhängigen App-Entwicklung},
    pdfcreator={Lukas Diedenhofen},
    pdfproducer={Lukas Diedenhofen},
    pdfkeywords={Dart} {Flutter} {ReactNative} {iOS} {Android} {Bachelorarbeit},
    colorlinks=false,
    linkcolor=red,
    citecolor=green,
    citebordercolor={0 1 0},
    filecolor=magenta,
    urlbordercolor={0 1 1},
    linktoc=all     %Verlinkungen im Inhaltsverzeichnis aktivieren oder speziell einstellen
}
\urlstyle{sf}

%pdftitle={Bachelorarbeit},pdfauthor={Lukas Diedenhofen},pdfcreator={Lukas Diedenhofen},pdfsubject={Evaluation von Frameworks zur plattformunabhängigen App-Entwicklung},

%-----------------------------------------------------------------------------
% Boxen auch zweispaltig
%-----------------------------------------------------------------------------
\usepackage{multicol}
\usepackage[most]{tcolorbox} % all but minted and documentation
%\usepackage[most,skins,listingsutf8]{tcolorbox}

% an dieser Stelle nach tcolorbox !!
\RequirePackage{setspace}
%\usepackages[onehalfspace]{setspace}
%%% [setspace]  %%%%%%%%%%%%%%%%%%%%%%%%%%%%%%%%%%%%%%%%%%%%%%
\onehalfspacing                                % Leerzeilen eineinhalbzeilig



\usepackage{environ}

% standard box
\newtcolorbox{tcbDefault}[1][]{
	colback=blue!5,
	colframe=blue!50,
	#1
}

% standard box with title
\newtcolorbox{tcbDefaultT}[2][]{
	coltitle=blue!50!black,
	colback=blue!5,
	colframe=blue!35,
	title={#2},
	%fonttitle=\bfseries,
	#1
}

%-----------------------------------------------------------------------------
% Listings with tcb
%-----------------------------------------------------------------------------

\newtcolorbox[auto counter,number within=section,]{tcbListingDefault}[2][]{
	enhanced,
	sharp corners=downhill,
	bicolor,
	arc=.5cm,
	top       = \tcboxedtitleheight,
	bottom    = 5mm,
	attach boxed title to top right={yshift=-\tcboxedtitleheight},
	boxed title style={
		size=small,
		colback=gray!20,
		colframe=gray!30,
		sharp corners=downhill,
		arc=.5cm,
		top=1mm,bottom=1mm,left=1mm,right=1mm
	},
	colframe  = gray!30,
	colback   = white,
	colbacklower=gray!5,
	coltitle = blue!50!black,
	fonttitle=\bfseries,
	leftupper=6mm,
	rightupper=0mm,
	leftlower=0mm,
	%boxsep=5mm, 
	middle=5mm,
	title=Listing~\thetcbcounter~ #2,
	#1
}

% with option side by side
\newtcolorbox[auto counter,number within=section,]{tcbListingDefaultSBS}[3][]{
	enhanced,
	sharp corners=downhill,
	bicolor,
	sidebyside,
	arc=.5cm,
	top       = \tcboxedtitleheight,
	bottom    = 0mm,
	attach boxed title to top right={yshift=-\tcboxedtitleheight},
	boxed title style={
		size=small,
		colback=colBackground,
		colframe=gray!30,
		sharp corners=downhill,
		arc=.5cm,
		top=1mm,bottom=1mm,left=1mm,right=1mm
	},
	colframe  = gray!30,
	colback   = white,
	colbacklower=gray!5,
	coltitle = blue!50!black,
	fonttitle=\bfseries,
	righthand width=#3,
	leftupper=6mm,
	rightupper=0mm,
	leftlower=0mm,
	boxsep=0mm, 
	middle=0mm,
	title=Listing~\thetcbcounter~ #2,
	#1
}


%\newtcbinputlisting[use counter=tcbListing]{\tcbinpListing}[3][]{%
\newtcbinputlisting[use counter from=tcbListingDefault]{\tcbinpListing}[3][]{%
	%\newtcbinputlisting[auto counter,number within=section]{\tcbinpListing}[3][]{%
	enhanced,
	bicolor,
	attach boxed title to top right={yshift=-\tcboxedtitleheight},
	boxed title style={
		size=small,
		colback=gray!20,
		colframe=gray!30,
		sharp corners=downhill,
		arc=.5cm,
		top=1mm,bottom=1mm,left=1mm,right=1mm
	},
	listing options={numbers=left,numberstyle=\tiny\color{red!75!black}},
	arc=.5cm,
	%left=6mm,      % nr ausserhalb Box
	boxsep=3mm,    % nr im Rand
	%fonttitle=\color{black}\itshape\ttfamily,
	fonttitle = \color{black}\bfseries,
	colframe  = gray!30,
	colback   = white,%gray!20,
	top       = \tcboxedtitleheight,
	bottom    = 0mm,
	sharp corners=downhill,
	listing file = {#2},
	%title        = Listing (\thetcbcounter) of \texttt{#2} #3,
	title        = Listing \thetcbcounter: #3,
	listing only,
	colbacklower=gray!5,
	breakable,
	#1
}

% Farben für die Programmierung festlegen:
\definecolor{VC_Blue}{RGB}{64,41,255}
\definecolor{VC_Green}{RGB}{42,139,15}
\definecolor{VC_BlueGrey}{RGB}{60,120,147}
\definecolor{VC_Brown}{RGB}{148,116,54}
\definecolor{VC_Blue2}{RGB}{12,113,197}
\definecolor{VC_Red}{RGB}{254,25,0}
\definecolor{VC_Green2}{RGB}{37,130,0}
\definecolor{VC_DarkBlue}{RGB}{12,24,128}
\definecolor{VC_Purpel}{RGB}{177,0,218}
\definecolor{VC_BrownRed}{RGB}{170,40,20}
\definecolor{VC_Dart_main}{RGB}{131,94,38}
\definecolor{UNIBW2}{RGB}{248,115,4}
\definecolor{UNIBW}{RGB}{236,111,0}
\definecolor{NumbersDart}{RGB}{15,130,101}
\colorlet{headerUNI}{UNIBW!60}
\colorlet{dunkelUNI}{gray!20}
%\colorlet{dunkelUNI}{UNIBW!25}
\colorlet{hellUNI}{white}
%\colorlet{hellUNI}{UNIBW!7}
\colorlet{dunkelZeile}{gray!20}


%-----------------------------------------------------------------------------
% Listings und pdf include
%-----------------------------------------------------------------------------
\usepackage{pdfpages}       % include pdf's 
\usepackage{listingsutf8}   % utf-8 encoded source files
\usepackage{arydshln}


% Quellcode: Eigenes Highlightening und Formatieren des Sourcecodes:
\RequirePackage{listings}

%-----------------------------------------------------------------------------
% Style for JavaScript und ReactNative
%-----------------------------------------------------------------------------

\definecolor{RN_Numbers}{RGB}{8,134,89}
\definecolor{RN_Key}{RGB}{0,0,255}
\definecolor{RN_Function}{RGB}{126,100,40}
\definecolor{VC_Brown}{RGB}{148,116,54}
\definecolor{VC_Blue2}{RGB}{12,113,197}
\definecolor{VC_Red}{RGB}{254,25,0}
\definecolor{VC_Green2}{RGB}{37,130,0}
\definecolor{VC_DarkBlue}{RGB}{12,24,128}
\definecolor{VC_Purpel}{RGB}{177,0,218}
\definecolor{VC_BrownRed}{RGB}{170,40,20}
\definecolor{VC_Dart_main}{RGB}{131,94,38}
\definecolor{UNIBW2}{RGB}{248,115,4}
\definecolor{UNIBW}{RGB}{236,111,0}
\definecolor{NumbersDart}{RGB}{15,130,101}

\lstdefinestyle{Java}{
    literate={0}{{\textcolor{RN_Numbers}{0}}}{1}%
                 {1}{{\textcolor{RN_Numbers}{1}}}{1}%
                 {2}{{\textcolor{RN_Numbers}{2}}}{1}%
                 {3}{{\textcolor{RN_Numbers}{3}}}{1}%
                 {4}{{\textcolor{RN_Numbers}{4}}}{1}%
                 {5}{{\textcolor{RN_Numbers}{5}}}{1}%
                 {6}{{\textcolor{RN_Numbers}{6}}}{1}%
                 {7}{{\textcolor{RN_Numbers}{7}}}{1}%
                 {8}{{\textcolor{RN_Numbers}{8}}}{1}%
                 {9}{{\textcolor{RN_Numbers}{9}}}{1}%
                 {.0}{{\textcolor{RN_Numbers}{.0}}}{2}% Following is to ensure that only periods
                 {.1}{{\textcolor{RN_Numbers}{.1}}}{2}% followed by a digit are changed.
                 {.2}{{\textcolor{RN_Numbers}{.2}}}{2}%
                 {.3}{{\textcolor{RN_Numbers}{.3}}}{2}%
                 {.4}{{\textcolor{RN_Numbers}{.4}}}{2}%
                 {.5}{{\textcolor{RN_Numbers}{.5}}}{2}%
                 {.6}{{\textcolor{RN_Numbers}{.6}}}{2}%
                 {.7}{{\textcolor{RN_Numbers}{.7}}}{2}%
                 {.8}{{\textcolor{RN_Numbers}{.8}}}{2}%
                 {.9}{{\textcolor{RN_Numbers}{.9}}}{2}%
                 ,
    language        =   Java,
    keywordstyle    =   \color{VC_Blue},
    commentstyle    =   \color{VC_Green},
    stringstyle     =   \itshape\color{VC_BrownRed},
    backgroundcolor =   \color{white},
    numberstyle     =   \small\ttfamily\color{VC_BlueGrey},
    basicstyle      =   \small\ttfamily\color{black},
    %basicstyle      =   \small\ttfamily\color{VC_DarkBlue},
    xleftmargin     =   25pt,
    emph            =   [1]{import, from, as, return, if, else, export, default, await},
    emphstyle       =   [1]\color{VC_Purpel},
    emph            =   [2]{*, const, function, false, null, =>, let, class, extends, async, true, this, super, constructor },
    emphstyle       =   [2]\color{VC_Blue},
    emph            =   [3]{[0-9]},
    emphstyle       =   [3]\color{VC_Red},
    emph            =   [4]{HelloWorldApp, View, Icon, Text, Image, ScrollView, Firebase, TouchableOpacity, SafeAreaView, StatusBar, StyleSheet, TextInput, Button, Alert},
    emphstyle       =   [4]\color{VC_BlueGrey},
    emph            =   [5]{style, name, size, source, contentContainerStyle, component, options, visible, onPress, disabled, author, barStyle, placeholder, keyboardType, underlineColorAndroid, onChangeText, title },
    emphstyle       =   [5]\color{VC_Red},
    emph            =   [6]{Platform, styles, },
    emphstyle       =   [6]\color{VC_Blue2},
    emph            =   [7]{count, state, props, buttonPlusFive, buttonPlusTwo, buttonPlusOne, buttonMinusFive, buttonMinusTwo, buttonMinusOne, textPlusOne, textPlusTwo, textPlusFive, textMinusOne, textMinusTwo, countContainer, text, counter, fixToText, container, OS, flex, justifyContent, paddingHorizontal, alignItems, backgroundColor, padding, width, height, shadowColor, shadowRadius, shadowOpacity, shadowOpacity, shadowOffset, elevation, color, fontSize, flexDirection, textReset, resetView, buttonReset, linksrechts, date, amount, category},
    emphstyle       =   [7]\color{VC_DarkBlue},
    emph            =   [8]{increment, setState, decrement, create, render, decrementTwo, decrementFive, incrementTwo, incrementFive, reset, setAddItemScreen, alert, addItem, onChangeAmount },
    emphstyle       =   [8]\color{RN_Function},
    frame           =   l,
}


%-----------------------------------------------------------------------------
% Style for Dart & flutter
%-----------------------------------------------------------------------------

\lstdefinestyle{Dart}{
    literate={0}{{\textcolor{NumbersDart}{0}}}{1}%
             {1}{{\textcolor{NumbersDart}{1}}}{1}%
             {2}{{\textcolor{NumbersDart}{2}}}{1}%
             {3}{{\textcolor{NumbersDart}{3}}}{1}%
             {4}{{\textcolor{NumbersDart}{4}}}{1}%
             {5}{{\textcolor{NumbersDart}{5}}}{1}%
             {6}{{\textcolor{NumbersDart}{6}}}{1}%
             {7}{{\textcolor{NumbersDart}{7}}}{1}%
             {8}{{\textcolor{NumbersDart}{8}}}{1}%
             {9}{{\textcolor{NumbersDart}{9}}}{1}%
             {.0}{{\textcolor{NumbersDart}{.0}}}{2}% Following is to ensure that only periods
             {.1}{{\textcolor{NumbersDart}{.1}}}{2}% followed by a digit are changed.
             {.2}{{\textcolor{NumbersDart}{.2}}}{2}%
             {.3}{{\textcolor{NumbersDart}{.3}}}{2}%
             {.4}{{\textcolor{NumbersDart}{.4}}}{2}%
             {.5}{{\textcolor{NumbersDart}{.5}}}{2}%
             {.6}{{\textcolor{NumbersDart}{.6}}}{2}%
             {.7}{{\textcolor{NumbersDart}{.7}}}{2}%
             {.8}{{\textcolor{NumbersDart}{.8}}}{2}%
             {.9}{{\textcolor{NumbersDart}{.9}}}{2}%
             ,
    language        =   c,
    keywordstyle    =   \color{VC_Blue},
    commentstyle    =   \color{VC_Green},
    stringstyle     =   \itshape\color{VC_BrownRed},
    backgroundcolor =   \color{white},
    numberstyle     =   \small\ttfamily\color{VC_BlueGrey},
    basicstyle      =   \small\ttfamily\color{black},
    xleftmargin     =   25pt,
    emph            =   [1]{return, if, for},
    emphstyle       =   [1]\color{VC_Purpel},
    emph            =   [2]{import, void, class, extends, @override, final, with, false, super, this, get, null, var, @required},
    emphstyle       =   [2]\color{VC_Blue},
    emph            =   [3]{StatefulWidget, Widget, Function, NewTransaction, MyApp, BuildContext, MaterialApp, Colors, ThemeData, AppBar, List,                                    Transaction, bool, double, String, Text, StatelessWidget, SafeArea, Column, Scaffold, Platform, Alignment, Radius, Container, TextStyle, TextAlign, EdgeInserts, RoundedRectangleBorder, BorderRadius, EdgeInsets, Ink, BoxDecoration, LinearGradient, Theme, BoxConstraints, RaisedButton, CupertinoPageScaffold, FloatingActionButton, Icon, Icons, FloatingActionButtonLocation, CupertinoButton, FontWeight, ElevatedButton},
    emphstyle       =   [3]\color{VC_BlueGrey},
    emph            =   [4]{main, build, runApp, light, copyWith, createState, initState, addObserver, print, dispose, toList, now, isAfter, where,                                 subtract, add, setState, _incrementCounter, of, all, elliptical, _startAddNewTransaction, _submitData, _excrementCounterFive},
    emphstyle       =   [4]\color{VC_Dart_main},
    emph            =   [5]{1,2,3,4,5,6,7,8,9,0},
    emphstyle       =   [5]\color{UNIBW},
    frame           =   l,
}


% Style für C
\lstdefinestyle{c}{
    commentstyle=\color{mGreen},
    keywordstyle=\color{magenta},
    stringstyle=\color{mPurple},
    breakatwhitespace=false,         
    breaklines=true,                 
    captionpos=b,                    
    keepspaces=false,                 
    numbers=left,                    
    numbersep=5pt,                  
    showspaces=false,                
    showstringspaces=false,
    showtabs=false,                  
    tabsize=2,
    language=C,
    emphstyle={\bfseries\color{red}},
}


% Einstellung des Default-Styles:
\lstset{ 
    basicstyle  = \small,              % Die Textgröße für den Quelltext
	numbers       = left,              % Platzierung Zeilennummern
	numberstyle   = \tiny\color{gray}, % Stil für die Seitennummern
	stepnumber    = 1,                 % Schritt zwischen nummerierten Zeilen.
	backgroundcolor= \color{white},    % Hintergrundfarbe für den Quelltext
	showspaces    = false,             % show spaces adding particular underscores
	showstringspaces = false,          % underline spaces within strings
	showtabs      = false,             % show tabs within strings adding particular underscores
	frame         = none,              % adds a frame around the code
	% lines, single, none
	rulecolor     = \color{black},     % if not set, the frame-color may be changed on
	% line-breaks within not-black text (e.g. comments (green here))
	tabsize       = 2,                 % Die Größe für einen Tabulator
	captionpos    = b,                 % Die Position der Überschrift. Hier b = Bottom
	breaklines    = true,              % Automatischer Zeilenumbruch
	breakatwhitespace=false,           % sets if automatic breaks only at whitespace
	title=\lstname,                    % show the filename of files 
	% WEGLASSEN damit lstlistings ohne caption weniger Platz brauchen
	keywordstyle  = \color{colKeyword},      % keyword style
	commentstyle  = \color{colComment}\ttfamily,   % comment style
	stringstyle   = \color{colString},     % string literal style
	escapeinside  = {\%*}{*)},         % if you want to add LaTeX within your code
	%linerange    = {1-13},            % Zeilenauswahl
	%linewidth    = 16cm               % = 0.9\linewidth, fix
	%xleftmargin  = 1.5cm,             % Einrücken von rechts
	%xrightmargin = 3.5cm,             % Einrücken von links
	%left         = 15pt,               % nr ausserhalb Box
	%boxsep       = 0pt,               % nr im Rand
	%boxrule      = 0pt,               % ohne Rand
	prebreak = \raisebox{0ex}[0ex][0ex]{\ensuremath{\hookleftarrow}},
	%	inputpath     = ./Source	         % default source input path
	% Achtung nicht inputpath verwenden sonst Konflikt mit tcblistings
}
\lstset{literate=
  {á}{{\'a}}1 {é}{{\'e}}1 {í}{{\'i}}1 {ó}{{\'o}}1 {ú}{{\'u}}1
  {Á}{{\'A}}1 {É}{{\'E}}1 {Í}{{\'I}}1 {Ó}{{\'O}}1 {Ú}{{\'U}}1
  {à}{{\`a}}1 {è}{{\`e}}1 {ì}{{\`i}}1 {ò}{{\`o}}1 {ù}{{\`u}}1
  {À}{{\`A}}1 {È}{{\'E}}1 {Ì}{{\`I}}1 {Ò}{{\`O}}1 {Ù}{{\`U}}1
  {ä}{{\"a}}1 {ë}{{\"e}}1 {ï}{{\"i}}1 {ö}{{\"o}}1 {ü}{{\"u}}1
  {Ä}{{\"A}}1 {Ë}{{\"E}}1 {Ï}{{\"I}}1 {Ö}{{\"O}}1 {Ü}{{\"U}}1
  {â}{{\^a}}1 {ê}{{\^e}}1 {î}{{\^i}}1 {ô}{{\^o}}1 {û}{{\^u}}1
  {Â}{{\^A}}1 {Ê}{{\^E}}1 {Î}{{\^I}}1 {Ô}{{\^O}}1 {Û}{{\^U}}1
  {ã}{{\~a}}1 {ẽ}{{\~e}}1 {ĩ}{{\~i}}1 {õ}{{\~o}}1 {ũ}{{\~u}}1
  {Ã}{{\~A}}1 {Ẽ}{{\~E}}1 {Ĩ}{{\~I}}1 {Õ}{{\~O}}1 {Ũ}{{\~U}}1
  {œ}{{\oe}}1 {Œ}{{\OE}}1 {æ}{{\ae}}1 {Æ}{{\AE}}1 {ß}{{\ss}}1
  {ű}{{\H{u}}}1 {Ű}{{\H{U}}}1 {ő}{{\H{o}}}1 {Ő}{{\H{O}}}1
  {ç}{{\c c}}1 {Ç}{{\c C}}1 {ø}{{\o}}1 {å}{{\r a}}1 {Å}{{\r A}}1
  {€}{{\euro}}1 {£}{{\pounds}}1 {«}{{\guillemotleft}}1
  {»}{{\guillemotright}}1 {ñ}{{\~n}}1 {Ñ}{{\~N}}1 {¿}{{?`}}1 {¡}{{!`}}1 
}


%----------------------------------------------------------------------------------------
% Inhaltsverzeichnis 
%----------------------------------------------------------------------------------------

%\usepackage{titletoc} % Manipulieren der Headings
%\contentsmargin{0cm} % Removes the default margin

% part style
%\titlecontents{part}
%[0cm]                       % Left indentation
%{\addvspace{20pt}\bfseries} % Spacing and font options
%{}
%{}
%{}

% Chapter Style
%\titlecontents{chapter}
%[1.25cm]                                                                        % Left indentation
%{\addvspace{10pt}\large\sffamily\bfseries}                                      % Spacing and font options
%{\color{black}\contentslabel[\large\thecontentslabel]{1.00cm}\color{black}}     % Formatting of numbered sections of this type
%{}                                                                              % Formatting of numberless sections of this type
%{\color{black}$\;$\small\titlerule*[.3cm]{ }$\;$\large\thecontentspage}         % filler and page number
%{\color{black}$\;$\small\titlerule*[.3cm]{.}$\;$\large\thecontentspage}

% Section Style
%\titlecontents{section}
%[2.5cm]                                                                         % Left indentation
%{\addvspace{3pt}\normalsize\sffamily}      %\bfseries                            % Spacing and font options
%{\contentslabel[\thecontentslabel]{1.25cm}}                                     % Formatting of numbered sections 
%{}                                                                              % Formatting of numberless sections of this type
%{\color{black}\small\titlerule*[.4cm]{.}$\;$\normalsize\thecontentspage}        % Formatting of the filler and the page number

% Subsection Style
%\titlecontents{subsection}
%[3.75cm]                                                                        % Left indentation
%{\addvspace{1pt}\sffamily\small}                                                % Spacing and font options
%{\contentslabel[\thecontentslabel]{1.25cm}}                                     % Formatting of numbered 
%{}                                                                              % Formatting of numberless sections of this type
%{\color{black}\small\titlerule*[.4cm]{.}$\;$\thecontentspage}                   % Formatting of the filler  and the page number


% · <-- Mittelpunkt
% – <-- Gedankenstrich


\usepackage{pgf-pie}
\usepackage{tikz}
\usepackage{pgfplots}
\pgfplotsset{width=15cm, height=8cm, compat=1.17}

\usepackage{bchart}