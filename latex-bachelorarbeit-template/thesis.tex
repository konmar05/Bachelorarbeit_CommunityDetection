%%%%%%%%%%%%%%%%%%%%%%%%%%%%%%%%%%%%%%%%%%%%%%%%%%%%%%%%%%%%%%%%%%%%%%%
%%  Der Einstiegspunkt des Latex-Dokuments                           %%
%%  Diese Datei führt alle Quelldokumente zusammen.                  %%
%%                                                                   %%
%%  Durch den Nutzer durchzuführende Änderungen                      %%
%%  sind entsprechend markiert.                                      %%
%%                                                                   %%
%%%%%%%%%%%%%%%%%%%%%%%%%%%%%%%%%%%%%%%%%%%%%%%%%%%%%%%%%%%%%%%%%%%%%%%
\documentclass[
paper=A4,            %a4paper, alle weiteren Papierformat einstellbar
11pt,                % Schriftgröße (12pt, 11pt (Standard))
BCOR=12mm,           % Bindekorrektur, bspw. 1 cm
headsepline,         % Trennline zum Seitenkopf  
footsepline,         % Trennline zum Seitenfuß
index=totoc,
listof=totoc,
bibliography=totoc,
]
{scrbook}

% 1) pdflatex main
% 2) makeindex.exe %.idx -s etti4.ist
% 3) biber main
% 4) pdflatex main x 2
%%%%%%%%%%%%%%%%%%%%%%%%%%%%%%%%%%%%%%%%%%%%%%%%%%%%%%%%%%%%%%%%%%%%%%%%%%%%%%
\usepackage{lipsum} % generate random text

%-----------------------------------------------------------------------------
% Deutsche Anpassungen
%-----------------------------------------------------------------------------
\usepackage[ngerman]{babel}      % deutsch content->Inhaltsverzeichnis ... 
\RequirePackage[babel]{microtype}
\usepackage[utf8]{inputenc}
%\usepackage[latin1]{inputenc}
\usepackage{inputenc}            % Umlaute
\usepackage{amssymb}             % Pfeile in math 
\usepackage{trfsigns}            % laplace ... Knoten
\RequirePackage{etex}
\usepackage{prettyref}
\usepackage{titleref}


% Font festlegen!
\RequirePackage{courier}								% Font: Schriftart Courier | Setzt Courier als True-Type-Schriftart (ttfamily)
% Change to Helvetica (close to Arial)
\RequirePackage[scaled]{helvet}
\RequirePackage[T1]{fontenc}

\renewcommand\familydefault{\sfdefault}

\usepackage{csquotes}
\usepackage[toc, acronym, nopostdot]{glossaries}
\usepackage{inconsolata}

\RequirePackage[absolute]{textpos}							    % Layout: Absolute Ausrichtung von Text
\RequirePackage[bottom]{footmisc}								% Fußnoten: Positionierung und Verhalten der Fußnoten ändern

\addto\captionsngerman{
  \renewcommand{\figurename}{Abbildung}
  \renewcommand{\tablename}{Tabelle}
  \renewcommand{\lstlistingname}{Quellcode}
}

%-----------------------------------------------------------------------------
% Grafik und Farben
%-----------------------------------------------------------------------------
%\usepackage{showframe} % <--------------------------------------------------- Ausblenden nicht vergessen! 
\usepackage{graphicx}            % Laden von Grafiken
\graphicspath{{./images/}}       % Pfad zu den Bildern
\usepackage{tikz}                % Grafik mit tex erzeugen
\usepackage{multirow}
% \documentclass[xcolor=table]{beamer}


% definierte Farben
% black, blue, brown, cyan, darkgray, gray, green, lightgray, lime, magenta,
% olive, orange, pink, purple, red, teal, violet, white, yellow
% Apricot 	  	  	  	Aquamarine          Bittersweet 	  	  	BlueGreen 
% BlueViolet   	  	    BrickRed            BurntOrange             CadetBlue
% CarnationPink         Cerulean 	  	  	CornflowerBlue          Dandelion
% DarkOrchid 	  	  	Emerald             ForestGreen 	  	  	Fuchsia
% Goldenrod 	  	  	GreenYellow
% JungleGreen 	  	  	Lavender            LimeGreen 	  	  	  	
% Mahogany 	  	  	  	Maroon              Melon 	  	  	  	    MidnightBlue
% Mulberry 	  	  	  	NavyBlue            OliveGreen 	  	  	  	
% OrangeRed 	  	  	Orchid              Peach 	  	  	  	    Periwinkle
% PineGreen 	  	   	Plum                ProcessBlue 	  	  	
% RawSienna 	  	   	RedOrange 	  	  	RedViolet
% Rhodamine 	  	  	RoyalBlue           RoyalPurple 	  	  	RubineRed
% Salmon 	  	  	  	SeaGreen            Sepia 	  	  	  	    SkyBlue
% SpringGreen 	  	  	Tan                 TealBlue 	  	  	  	Thistle
% Turquoise 	  	  	VioletRed 	  	  	  	
% WildStrawberry 	  	YellowGreen 	  	YellowOrange

% RGB define new colors
\definecolor{dkgreen}{rgb}{0,0.6,0}
\definecolor{mauve}{rgb}{0.58,0,0.82}
\definecolor{darkblue}{rgb}{0.1,0,1.0}
\definecolor{darkgreen}{rgb}{0.1,0.8,0.2}
\definecolor{lightblue}{rgb}{0.8,0.85,1}
\definecolor{blue2}{RGB}{0,170,255}
\definecolor{pale turquoise}{rgb}{0.686,0.933,0.933}
\definecolor{light-gray}{gray}{0.95}
\definecolor{ettiOrange}{HTML}{ED6D00}

% Programmcode Farben
\definecolor{mGreen}{rgb}{0,0.6,0}
\definecolor{mGray}{rgb}{0.5,0.5,0.5}
\definecolor{mPurple}{rgb}{0.58,0,0.82}
\definecolor{backgroundColour}{RGB}{240,248,255}

% reuse and mix colors
\colorlet{vlightgray}{lightgray!20}

% mapping - nur die folgenden Farben verwenden
\colorlet{colBackground}{vlightgray}
\colorlet{colPrimary}{blue2}
\colorlet{colSuccess}{green}
\colorlet{colDanger}{red}
\colorlet{colWarning}{magenta!80}
\colorlet{colGrid}{lightgray!40}

% defaults for listings
\colorlet{colKeyword}{magenta}
\colorlet{colComment}{dkgreen}
\colorlet{colString}{mauve}

% defaults for Inhaltsverzeichnis und chapterthumb
\colorlet{colScriptContent}{black}
\colorlet{colScriptCT}{lightblue}



%-----------------------------------------------------------------------------
% Layout
%-----------------------------------------------------------------------------
\parindent0cm
\usepackage{enumitem} % kleinere Abstände

\usepackage{geometry}

			
%\geometry{includehead,includefoot,inner=2.5cm,outer=1.5cm,top=1.5cm,bottom=1.0cm}
\geometry{includehead,includefoot, inner=2.25cm,outer=2.25cm,top=1.5cm,bottom=1.5cm}
\setlength{\footheight}{20.4pt}
\setlength{\headheight}{20.4pt}

\usepackage{siunitx}                 %Mathe Komma einstellen! 
\sisetup{
    detect-family = true,
    %detect-inline-family = math,
    output-decimal-marker = {,}
}


\usepackage{needspace} % Seitenumbruch falls notwendiger Platz nicht mehr vorhanden


\usepackage{rotating}

\usepackage{scrlayer-scrpage}
\pagestyle{scrheadings}
\RequirePackage{xspace}     % Intelligentes Leerzeichen

%-----------------------------------------------------------------------------
% Kopf- und Fußzeile manipulieren
%-----------------------------------------------------------------------------

\renewcommand{\headfont}{\normalfont\sffamily\itshape}          % Kolumnentitel serifenlos
\renewcommand{\pnumfont}{\normalfont\rmfamily}         % Seitennummern typewriter


%-----------------------------------------------------------------------------
% Tabellen
%-----------------------------------------------------------------------------
\usepackage{colortbl}  % Farbige Tabellen
%\usepackage{longtable} % Tabellen die über mehr als eine Seite gehen
\usepackage{booktabs} 
\usepackage{tabularx}
\RequirePackage{tabu}	% Listen:   Erweiterte Listenformatierung tabu Umgebung
\setlength\arrayrulewidth{0.5pt}
\usepackage{makecell}       %Zeilenumbrüche in Tabellen einfügen


% ZUM EINFÜGEN VON BILDERN in TABELLEN
% neuer Befehl: \includegraphicstotab[..]{..}
% Verwendung analog wie \includegraphics
\newlength{\myx} % Variable zum Speichern der Bildbreite
\newlength{\myy} % Variable zum Speichern der Bildhöhe

\newcommand\includegraphicstotab[2][\relax]{%
	% Abspeichern der Bildabmessungen
	\settowidth{\myx}{\includegraphics[{#1}]{#2}}%
	\settoheight{\myy}{\includegraphics[{#1}]{#2}}%
	% das eigentliche Einfügen
	\parbox[c][1.1\myy][c]{\myx}{%
		\includegraphics[{#1}]{#2}}%
}% Ende neuer Befehl


%-----------------------------------------------------------------------------
% Literaturverzeichnis
%-----------------------------------------------------------------------------
\usepackage[
backend=biber,
style=numeric, % style=alphabetic authoryear numeric,
citestyle=numeric,sorting=none,sortcites=true,autopunct=true,
hyperref=true,abbreviate=false,backref=true,block=ragged,
]{biblatex}
\addbibresource{./Literatur.bib} % Literaturdatenbank
\defbibheading{bibempty}{}


%-----------------------------------------------------------------------------
% Index
%-----------------------------------------------------------------------------
\usepackage{makeidx} % Für das Erstellen von einem Index
\makeindex


%-----------------------------------------------------------------------------
% Datum und Zeit
%-----------------------------------------------------------------------------
\usepackage{scrtime}
%-----------------------------------------------------------------------------
% Text
%-----------------------------------------------------------------------------
%\usepackage{microtype} % besserer Wörterabstand und weniger Textbreitenfehler

\RequirePackage[T1]{fontenc}
\RequirePackage{lmodern}
\usepackage{textcomp}
%\usepackage[scale=0.98,ttdefault]{AnonymousPro}
\usepackage{times}

\usepackage{exscale}   % schöne Formeln in PDFs

%-----------------------------------------------------------------------------
% Verlinkungen im Text
%-----------------------------------------------------------------------------
\usepackage[pdftex,bookmarks=true]{hyperref}
\hypersetup{
    pdfborder={0 0 0}, %Es wird kein Rahmen um irgendeine Verlinkung angezeigt, Verlinkungen bleiben aber bestehen
    bookmarksopen=true,
    bookmarksnumbered=true,
    pdftoolbar=true,
    pdfmenubar=true,
    pdffitwindow=true,
    pdfstartview={FitBH},
    pdftitle={Bachelorarbeit},
    pdfauthor={Lukas Diedenhofen},
    pdfsubject={Evaluation von Frameworks zur plattformunabhängigen App-Entwicklung},
    pdfcreator={Lukas Diedenhofen},
    pdfproducer={Lukas Diedenhofen},
    pdfkeywords={Dart} {Flutter} {ReactNative} {iOS} {Android} {Bachelorarbeit},
    colorlinks=false,
    linkcolor=red,
    citecolor=green,
    citebordercolor={0 1 0},
    filecolor=magenta,
    urlbordercolor={0 1 1},
    linktoc=all     %Verlinkungen im Inhaltsverzeichnis aktivieren oder speziell einstellen
}
\urlstyle{sf}

%pdftitle={Bachelorarbeit},pdfauthor={Lukas Diedenhofen},pdfcreator={Lukas Diedenhofen},pdfsubject={Evaluation von Frameworks zur plattformunabhängigen App-Entwicklung},

%-----------------------------------------------------------------------------
% Boxen auch zweispaltig
%-----------------------------------------------------------------------------
\usepackage{multicol}
\usepackage[most]{tcolorbox} % all but minted and documentation
%\usepackage[most,skins,listingsutf8]{tcolorbox}

% an dieser Stelle nach tcolorbox !!
\RequirePackage{setspace}
%\usepackages[onehalfspace]{setspace}
%%% [setspace]  %%%%%%%%%%%%%%%%%%%%%%%%%%%%%%%%%%%%%%%%%%%%%%
\onehalfspacing                                % Leerzeilen eineinhalbzeilig



\usepackage{environ}

% standard box
\newtcolorbox{tcbDefault}[1][]{
	colback=blue!5,
	colframe=blue!50,
	#1
}

% standard box with title
\newtcolorbox{tcbDefaultT}[2][]{
	coltitle=blue!50!black,
	colback=blue!5,
	colframe=blue!35,
	title={#2},
	%fonttitle=\bfseries,
	#1
}

%-----------------------------------------------------------------------------
% Listings with tcb
%-----------------------------------------------------------------------------

\newtcolorbox[auto counter,number within=section,]{tcbListingDefault}[2][]{
	enhanced,
	sharp corners=downhill,
	bicolor,
	arc=.5cm,
	top       = \tcboxedtitleheight,
	bottom    = 5mm,
	attach boxed title to top right={yshift=-\tcboxedtitleheight},
	boxed title style={
		size=small,
		colback=gray!20,
		colframe=gray!30,
		sharp corners=downhill,
		arc=.5cm,
		top=1mm,bottom=1mm,left=1mm,right=1mm
	},
	colframe  = gray!30,
	colback   = white,
	colbacklower=gray!5,
	coltitle = blue!50!black,
	fonttitle=\bfseries,
	leftupper=6mm,
	rightupper=0mm,
	leftlower=0mm,
	%boxsep=5mm, 
	middle=5mm,
	title=Listing~\thetcbcounter~ #2,
	#1
}

% with option side by side
\newtcolorbox[auto counter,number within=section,]{tcbListingDefaultSBS}[3][]{
	enhanced,
	sharp corners=downhill,
	bicolor,
	sidebyside,
	arc=.5cm,
	top       = \tcboxedtitleheight,
	bottom    = 0mm,
	attach boxed title to top right={yshift=-\tcboxedtitleheight},
	boxed title style={
		size=small,
		colback=colBackground,
		colframe=gray!30,
		sharp corners=downhill,
		arc=.5cm,
		top=1mm,bottom=1mm,left=1mm,right=1mm
	},
	colframe  = gray!30,
	colback   = white,
	colbacklower=gray!5,
	coltitle = blue!50!black,
	fonttitle=\bfseries,
	righthand width=#3,
	leftupper=6mm,
	rightupper=0mm,
	leftlower=0mm,
	boxsep=0mm, 
	middle=0mm,
	title=Listing~\thetcbcounter~ #2,
	#1
}


%\newtcbinputlisting[use counter=tcbListing]{\tcbinpListing}[3][]{%
\newtcbinputlisting[use counter from=tcbListingDefault]{\tcbinpListing}[3][]{%
	%\newtcbinputlisting[auto counter,number within=section]{\tcbinpListing}[3][]{%
	enhanced,
	bicolor,
	attach boxed title to top right={yshift=-\tcboxedtitleheight},
	boxed title style={
		size=small,
		colback=gray!20,
		colframe=gray!30,
		sharp corners=downhill,
		arc=.5cm,
		top=1mm,bottom=1mm,left=1mm,right=1mm
	},
	listing options={numbers=left,numberstyle=\tiny\color{red!75!black}},
	arc=.5cm,
	%left=6mm,      % nr ausserhalb Box
	boxsep=3mm,    % nr im Rand
	%fonttitle=\color{black}\itshape\ttfamily,
	fonttitle = \color{black}\bfseries,
	colframe  = gray!30,
	colback   = white,%gray!20,
	top       = \tcboxedtitleheight,
	bottom    = 0mm,
	sharp corners=downhill,
	listing file = {#2},
	%title        = Listing (\thetcbcounter) of \texttt{#2} #3,
	title        = Listing \thetcbcounter: #3,
	listing only,
	colbacklower=gray!5,
	breakable,
	#1
}

% Farben für die Programmierung festlegen:
\definecolor{VC_Blue}{RGB}{64,41,255}
\definecolor{VC_Green}{RGB}{42,139,15}
\definecolor{VC_BlueGrey}{RGB}{60,120,147}
\definecolor{VC_Brown}{RGB}{148,116,54}
\definecolor{VC_Blue2}{RGB}{12,113,197}
\definecolor{VC_Red}{RGB}{254,25,0}
\definecolor{VC_Green2}{RGB}{37,130,0}
\definecolor{VC_DarkBlue}{RGB}{12,24,128}
\definecolor{VC_Purpel}{RGB}{177,0,218}
\definecolor{VC_BrownRed}{RGB}{170,40,20}
\definecolor{VC_Dart_main}{RGB}{131,94,38}
\definecolor{UNIBW2}{RGB}{248,115,4}
\definecolor{UNIBW}{RGB}{236,111,0}
\definecolor{NumbersDart}{RGB}{15,130,101}
\colorlet{headerUNI}{UNIBW!60}
\colorlet{dunkelUNI}{gray!20}
%\colorlet{dunkelUNI}{UNIBW!25}
\colorlet{hellUNI}{white}
%\colorlet{hellUNI}{UNIBW!7}
\colorlet{dunkelZeile}{gray!20}


%-----------------------------------------------------------------------------
% Listings und pdf include
%-----------------------------------------------------------------------------
\usepackage{pdfpages}       % include pdf's 
\usepackage{listingsutf8}   % utf-8 encoded source files
\usepackage{arydshln}


% Quellcode: Eigenes Highlightening und Formatieren des Sourcecodes:
\RequirePackage{listings}

%-----------------------------------------------------------------------------
% Style for JavaScript und ReactNative
%-----------------------------------------------------------------------------

\definecolor{RN_Numbers}{RGB}{8,134,89}
\definecolor{RN_Key}{RGB}{0,0,255}
\definecolor{RN_Function}{RGB}{126,100,40}
\definecolor{VC_Brown}{RGB}{148,116,54}
\definecolor{VC_Blue2}{RGB}{12,113,197}
\definecolor{VC_Red}{RGB}{254,25,0}
\definecolor{VC_Green2}{RGB}{37,130,0}
\definecolor{VC_DarkBlue}{RGB}{12,24,128}
\definecolor{VC_Purpel}{RGB}{177,0,218}
\definecolor{VC_BrownRed}{RGB}{170,40,20}
\definecolor{VC_Dart_main}{RGB}{131,94,38}
\definecolor{UNIBW2}{RGB}{248,115,4}
\definecolor{UNIBW}{RGB}{236,111,0}
\definecolor{NumbersDart}{RGB}{15,130,101}

\lstdefinestyle{Java}{
    literate={0}{{\textcolor{RN_Numbers}{0}}}{1}%
                 {1}{{\textcolor{RN_Numbers}{1}}}{1}%
                 {2}{{\textcolor{RN_Numbers}{2}}}{1}%
                 {3}{{\textcolor{RN_Numbers}{3}}}{1}%
                 {4}{{\textcolor{RN_Numbers}{4}}}{1}%
                 {5}{{\textcolor{RN_Numbers}{5}}}{1}%
                 {6}{{\textcolor{RN_Numbers}{6}}}{1}%
                 {7}{{\textcolor{RN_Numbers}{7}}}{1}%
                 {8}{{\textcolor{RN_Numbers}{8}}}{1}%
                 {9}{{\textcolor{RN_Numbers}{9}}}{1}%
                 {.0}{{\textcolor{RN_Numbers}{.0}}}{2}% Following is to ensure that only periods
                 {.1}{{\textcolor{RN_Numbers}{.1}}}{2}% followed by a digit are changed.
                 {.2}{{\textcolor{RN_Numbers}{.2}}}{2}%
                 {.3}{{\textcolor{RN_Numbers}{.3}}}{2}%
                 {.4}{{\textcolor{RN_Numbers}{.4}}}{2}%
                 {.5}{{\textcolor{RN_Numbers}{.5}}}{2}%
                 {.6}{{\textcolor{RN_Numbers}{.6}}}{2}%
                 {.7}{{\textcolor{RN_Numbers}{.7}}}{2}%
                 {.8}{{\textcolor{RN_Numbers}{.8}}}{2}%
                 {.9}{{\textcolor{RN_Numbers}{.9}}}{2}%
                 ,
    language        =   Java,
    keywordstyle    =   \color{VC_Blue},
    commentstyle    =   \color{VC_Green},
    stringstyle     =   \itshape\color{VC_BrownRed},
    backgroundcolor =   \color{white},
    numberstyle     =   \small\ttfamily\color{VC_BlueGrey},
    basicstyle      =   \small\ttfamily\color{black},
    %basicstyle      =   \small\ttfamily\color{VC_DarkBlue},
    xleftmargin     =   25pt,
    emph            =   [1]{import, from, as, return, if, else, export, default, await},
    emphstyle       =   [1]\color{VC_Purpel},
    emph            =   [2]{*, const, function, false, null, =>, let, class, extends, async, true, this, super, constructor },
    emphstyle       =   [2]\color{VC_Blue},
    emph            =   [3]{[0-9]},
    emphstyle       =   [3]\color{VC_Red},
    emph            =   [4]{HelloWorldApp, View, Icon, Text, Image, ScrollView, Firebase, TouchableOpacity, SafeAreaView, StatusBar, StyleSheet, TextInput, Button, Alert},
    emphstyle       =   [4]\color{VC_BlueGrey},
    emph            =   [5]{style, name, size, source, contentContainerStyle, component, options, visible, onPress, disabled, author, barStyle, placeholder, keyboardType, underlineColorAndroid, onChangeText, title },
    emphstyle       =   [5]\color{VC_Red},
    emph            =   [6]{Platform, styles, },
    emphstyle       =   [6]\color{VC_Blue2},
    emph            =   [7]{count, state, props, buttonPlusFive, buttonPlusTwo, buttonPlusOne, buttonMinusFive, buttonMinusTwo, buttonMinusOne, textPlusOne, textPlusTwo, textPlusFive, textMinusOne, textMinusTwo, countContainer, text, counter, fixToText, container, OS, flex, justifyContent, paddingHorizontal, alignItems, backgroundColor, padding, width, height, shadowColor, shadowRadius, shadowOpacity, shadowOpacity, shadowOffset, elevation, color, fontSize, flexDirection, textReset, resetView, buttonReset, linksrechts, date, amount, category},
    emphstyle       =   [7]\color{VC_DarkBlue},
    emph            =   [8]{increment, setState, decrement, create, render, decrementTwo, decrementFive, incrementTwo, incrementFive, reset, setAddItemScreen, alert, addItem, onChangeAmount },
    emphstyle       =   [8]\color{RN_Function},
    frame           =   l,
}


%-----------------------------------------------------------------------------
% Style for Dart & flutter
%-----------------------------------------------------------------------------

\lstdefinestyle{Dart}{
    literate={0}{{\textcolor{NumbersDart}{0}}}{1}%
             {1}{{\textcolor{NumbersDart}{1}}}{1}%
             {2}{{\textcolor{NumbersDart}{2}}}{1}%
             {3}{{\textcolor{NumbersDart}{3}}}{1}%
             {4}{{\textcolor{NumbersDart}{4}}}{1}%
             {5}{{\textcolor{NumbersDart}{5}}}{1}%
             {6}{{\textcolor{NumbersDart}{6}}}{1}%
             {7}{{\textcolor{NumbersDart}{7}}}{1}%
             {8}{{\textcolor{NumbersDart}{8}}}{1}%
             {9}{{\textcolor{NumbersDart}{9}}}{1}%
             {.0}{{\textcolor{NumbersDart}{.0}}}{2}% Following is to ensure that only periods
             {.1}{{\textcolor{NumbersDart}{.1}}}{2}% followed by a digit are changed.
             {.2}{{\textcolor{NumbersDart}{.2}}}{2}%
             {.3}{{\textcolor{NumbersDart}{.3}}}{2}%
             {.4}{{\textcolor{NumbersDart}{.4}}}{2}%
             {.5}{{\textcolor{NumbersDart}{.5}}}{2}%
             {.6}{{\textcolor{NumbersDart}{.6}}}{2}%
             {.7}{{\textcolor{NumbersDart}{.7}}}{2}%
             {.8}{{\textcolor{NumbersDart}{.8}}}{2}%
             {.9}{{\textcolor{NumbersDart}{.9}}}{2}%
             ,
    language        =   c,
    keywordstyle    =   \color{VC_Blue},
    commentstyle    =   \color{VC_Green},
    stringstyle     =   \itshape\color{VC_BrownRed},
    backgroundcolor =   \color{white},
    numberstyle     =   \small\ttfamily\color{VC_BlueGrey},
    basicstyle      =   \small\ttfamily\color{black},
    xleftmargin     =   25pt,
    emph            =   [1]{return, if, for},
    emphstyle       =   [1]\color{VC_Purpel},
    emph            =   [2]{import, void, class, extends, @override, final, with, false, super, this, get, null, var, @required},
    emphstyle       =   [2]\color{VC_Blue},
    emph            =   [3]{StatefulWidget, Widget, Function, NewTransaction, MyApp, BuildContext, MaterialApp, Colors, ThemeData, AppBar, List,                                    Transaction, bool, double, String, Text, StatelessWidget, SafeArea, Column, Scaffold, Platform, Alignment, Radius, Container, TextStyle, TextAlign, EdgeInserts, RoundedRectangleBorder, BorderRadius, EdgeInsets, Ink, BoxDecoration, LinearGradient, Theme, BoxConstraints, RaisedButton, CupertinoPageScaffold, FloatingActionButton, Icon, Icons, FloatingActionButtonLocation, CupertinoButton, FontWeight, ElevatedButton},
    emphstyle       =   [3]\color{VC_BlueGrey},
    emph            =   [4]{main, build, runApp, light, copyWith, createState, initState, addObserver, print, dispose, toList, now, isAfter, where,                                 subtract, add, setState, _incrementCounter, of, all, elliptical, _startAddNewTransaction, _submitData, _excrementCounterFive},
    emphstyle       =   [4]\color{VC_Dart_main},
    emph            =   [5]{1,2,3,4,5,6,7,8,9,0},
    emphstyle       =   [5]\color{UNIBW},
    frame           =   l,
}


% Style für C
\lstdefinestyle{c}{
    commentstyle=\color{mGreen},
    keywordstyle=\color{magenta},
    stringstyle=\color{mPurple},
    breakatwhitespace=false,         
    breaklines=true,                 
    captionpos=b,                    
    keepspaces=false,                 
    numbers=left,                    
    numbersep=5pt,                  
    showspaces=false,                
    showstringspaces=false,
    showtabs=false,                  
    tabsize=2,
    language=C,
    emphstyle={\bfseries\color{red}},
}


% Einstellung des Default-Styles:
\lstset{ 
    basicstyle  = \small,              % Die Textgröße für den Quelltext
	numbers       = left,              % Platzierung Zeilennummern
	numberstyle   = \tiny\color{gray}, % Stil für die Seitennummern
	stepnumber    = 1,                 % Schritt zwischen nummerierten Zeilen.
	backgroundcolor= \color{white},    % Hintergrundfarbe für den Quelltext
	showspaces    = false,             % show spaces adding particular underscores
	showstringspaces = false,          % underline spaces within strings
	showtabs      = false,             % show tabs within strings adding particular underscores
	frame         = none,              % adds a frame around the code
	% lines, single, none
	rulecolor     = \color{black},     % if not set, the frame-color may be changed on
	% line-breaks within not-black text (e.g. comments (green here))
	tabsize       = 2,                 % Die Größe für einen Tabulator
	captionpos    = b,                 % Die Position der Überschrift. Hier b = Bottom
	breaklines    = true,              % Automatischer Zeilenumbruch
	breakatwhitespace=false,           % sets if automatic breaks only at whitespace
	title=\lstname,                    % show the filename of files 
	% WEGLASSEN damit lstlistings ohne caption weniger Platz brauchen
	keywordstyle  = \color{colKeyword},      % keyword style
	commentstyle  = \color{colComment}\ttfamily,   % comment style
	stringstyle   = \color{colString},     % string literal style
	escapeinside  = {\%*}{*)},         % if you want to add LaTeX within your code
	%linerange    = {1-13},            % Zeilenauswahl
	%linewidth    = 16cm               % = 0.9\linewidth, fix
	%xleftmargin  = 1.5cm,             % Einrücken von rechts
	%xrightmargin = 3.5cm,             % Einrücken von links
	%left         = 15pt,               % nr ausserhalb Box
	%boxsep       = 0pt,               % nr im Rand
	%boxrule      = 0pt,               % ohne Rand
	prebreak = \raisebox{0ex}[0ex][0ex]{\ensuremath{\hookleftarrow}},
	%	inputpath     = ./Source	         % default source input path
	% Achtung nicht inputpath verwenden sonst Konflikt mit tcblistings
}
\lstset{literate=
  {á}{{\'a}}1 {é}{{\'e}}1 {í}{{\'i}}1 {ó}{{\'o}}1 {ú}{{\'u}}1
  {Á}{{\'A}}1 {É}{{\'E}}1 {Í}{{\'I}}1 {Ó}{{\'O}}1 {Ú}{{\'U}}1
  {à}{{\`a}}1 {è}{{\`e}}1 {ì}{{\`i}}1 {ò}{{\`o}}1 {ù}{{\`u}}1
  {À}{{\`A}}1 {È}{{\'E}}1 {Ì}{{\`I}}1 {Ò}{{\`O}}1 {Ù}{{\`U}}1
  {ä}{{\"a}}1 {ë}{{\"e}}1 {ï}{{\"i}}1 {ö}{{\"o}}1 {ü}{{\"u}}1
  {Ä}{{\"A}}1 {Ë}{{\"E}}1 {Ï}{{\"I}}1 {Ö}{{\"O}}1 {Ü}{{\"U}}1
  {â}{{\^a}}1 {ê}{{\^e}}1 {î}{{\^i}}1 {ô}{{\^o}}1 {û}{{\^u}}1
  {Â}{{\^A}}1 {Ê}{{\^E}}1 {Î}{{\^I}}1 {Ô}{{\^O}}1 {Û}{{\^U}}1
  {ã}{{\~a}}1 {ẽ}{{\~e}}1 {ĩ}{{\~i}}1 {õ}{{\~o}}1 {ũ}{{\~u}}1
  {Ã}{{\~A}}1 {Ẽ}{{\~E}}1 {Ĩ}{{\~I}}1 {Õ}{{\~O}}1 {Ũ}{{\~U}}1
  {œ}{{\oe}}1 {Œ}{{\OE}}1 {æ}{{\ae}}1 {Æ}{{\AE}}1 {ß}{{\ss}}1
  {ű}{{\H{u}}}1 {Ű}{{\H{U}}}1 {ő}{{\H{o}}}1 {Ő}{{\H{O}}}1
  {ç}{{\c c}}1 {Ç}{{\c C}}1 {ø}{{\o}}1 {å}{{\r a}}1 {Å}{{\r A}}1
  {€}{{\euro}}1 {£}{{\pounds}}1 {«}{{\guillemotleft}}1
  {»}{{\guillemotright}}1 {ñ}{{\~n}}1 {Ñ}{{\~N}}1 {¿}{{?`}}1 {¡}{{!`}}1 
}


%----------------------------------------------------------------------------------------
% Inhaltsverzeichnis 
%----------------------------------------------------------------------------------------

%\usepackage{titletoc} % Manipulieren der Headings
%\contentsmargin{0cm} % Removes the default margin

% part style
%\titlecontents{part}
%[0cm]                       % Left indentation
%{\addvspace{20pt}\bfseries} % Spacing and font options
%{}
%{}
%{}

% Chapter Style
%\titlecontents{chapter}
%[1.25cm]                                                                        % Left indentation
%{\addvspace{10pt}\large\sffamily\bfseries}                                      % Spacing and font options
%{\color{black}\contentslabel[\large\thecontentslabel]{1.00cm}\color{black}}     % Formatting of numbered sections of this type
%{}                                                                              % Formatting of numberless sections of this type
%{\color{black}$\;$\small\titlerule*[.3cm]{ }$\;$\large\thecontentspage}         % filler and page number
%{\color{black}$\;$\small\titlerule*[.3cm]{.}$\;$\large\thecontentspage}

% Section Style
%\titlecontents{section}
%[2.5cm]                                                                         % Left indentation
%{\addvspace{3pt}\normalsize\sffamily}      %\bfseries                            % Spacing and font options
%{\contentslabel[\thecontentslabel]{1.25cm}}                                     % Formatting of numbered sections 
%{}                                                                              % Formatting of numberless sections of this type
%{\color{black}\small\titlerule*[.4cm]{.}$\;$\normalsize\thecontentspage}        % Formatting of the filler and the page number

% Subsection Style
%\titlecontents{subsection}
%[3.75cm]                                                                        % Left indentation
%{\addvspace{1pt}\sffamily\small}                                                % Spacing and font options
%{\contentslabel[\thecontentslabel]{1.25cm}}                                     % Formatting of numbered 
%{}                                                                              % Formatting of numberless sections of this type
%{\color{black}\small\titlerule*[.4cm]{.}$\;$\thecontentspage}                   % Formatting of the filler  and the page number


% · <-- Mittelpunkt
% – <-- Gedankenstrich


\usepackage{pgf-pie}
\usepackage{tikz}
\usepackage{pgfplots}
\pgfplotsset{width=15cm, height=8cm, compat=1.17}

\usepackage{bchart}          % Laden aller notwendigen Pakte und Definitionen	
\usepackage{tabu} 


%% Packages für Grafiken & Abbildungen 
\usepackage{subfigure} %%Teilabbildungen in einer Abbildung
\usepackage{caption}

\newenvironment{abstract}{}{}
\usepackage{abstract}


\RequirePackage{float}							% Gleitobjekte(table,figure): Erweitertung
\RequirePackage{placeins}						% Gleitobjekte(table,figure): Position einschränken

%% Sonstige Pakete 
\usepackage{eurosym}                            % Euro Symbol
\usepackage{dsfont}                             % Fonts für Zahlenbereiche


%% Abkürzungen
%%%%%%%%%%%%%%%%%%%%%%%%%%%%%%%%%%%%%%%%
%%Acronyms%%%%%%%%%%%%%%%%%%%%%%%%%%%%%%
%%%%%%%%%%%%%%%%%%%%%%%%%%%%%%%%%%%%%%%%
\newacronym{ba}{BA}{Bachelorarbeit}
\newacronym{etti}{ETTI}{Elektrotechnik und technische Informatik}


%%%%%%%%%%%%%%%%%%%%%%%%%%%%%%%%%%%%%%%%
%%Definitions%%%%%%%%%%%%%%%%%%%%%%%%%%%
%%%%%%%%%%%%%%%%%%%%%%%%%%%%%%%%%%%%%%%%
\newglossaryentry{latex}
{
        name=latex,
        description={Is a mark up language specially suited for 
scientific documents}
}
\makeglossaries


\lstdefinelanguage{NOSTYLE}{
  alsodigit = {-},
  keywords = {}
}

\setlength{\parindent}{0em}

\usepackage{pgf-pie}
\usepackage{pgfgantt}
\usepackage{blindtext}
\usepackage{verbatim}



%%%%%%%%%%%%%%%%%%%%%%%%%%%%%%%%%%%%%%%%%%%%%%%%%%%%%%%%%%%%%%%%%%%%%%%%%%%
%% Die folgenden Variablen sind durch den Nutzer anzupassen.             %%
%% Sie werden in den entsprechenden Dokumenten automatisch übernommen;   %%
%% Änderungen in den Dokumenten sind dann nicht mehr nötig.              %%
%%%%%%%%%%%%%%%%%%%%%%%%%%%%%%%%%%%%%%%%%%%%%%%%%%%%%%%%%%%%%%%%%%%%%%%%%%%
\def\theCourseOfStudies{Technische Informatik und Kommunikationstechnik}
\def\theCertificate{Bachelorarbeit}
\def\theTitle{Bewertung von Community Detection Algorithmen mit der CDlib-Bibliothek}
\def\theSubTitle{}
\def\theAuthor{Markus Konietzka}
\def\theAdviserA{Prof. Dr. rer. nat. Andrea Baumann}
\def\theAdviserB{~}
\def\thePublicationDate{30.06.2023}
\def\thePublicationMY{Juni 2023}
%%%%%%%%%%%%%%%%%%%%%%%%%%%%%%%%%%%%%%%%%%%%%%%%%%%%%%%%%%%%%%%%%%%%%%%%%%%
%%%%%%%%%%%%%%%%%%%%%%%%%%%%%%%%%%%%%%%%%%%%%%%%%%%%%%%%%%%%%%%%%%%%%%%%%%%

\DeclareMathOperator{\ceil}{ceil}
\DeclareMathOperator{\floor}{floor}

\newcounter{savepage}
\begin{document}


%Caption nach unten setzen
\captionsetup[figure]{skip=10pt}
%\captionsetup[table]{skip=10pt}
\captionsetup[lstlisting]{skip=10pt}

\pagenumbering{Roman}

%% Titelseite
%%%%%%%%%%%%%%%%%%%%%%%%%%%%%%%%%%%%%%%%%%%%%%%%%%%%%%%%%%%%%%%%%%%%%%%%%%%
%% Die Titelseite des Dokuments.                                         %%
%% In diesem Dokument sind keine Änderungen durch den Nutzer             %%
%% notwendig; Die Variablen werden aus der thesis.tex übernommen.        %%
%%%%%%%%%%%%%%%%%%%%%%%%%%%%%%%%%%%%%%%%%%%%%%%%%%%%%%%%%%%%%%%%%%%%%%%%%%%
\begin{titlepage}

\setlength{\topmargin}{0pt}

\begin{center}

\begin{center}
  \includegraphics[width=12cm]{images/logos/UniBwMSignet.pdf}
\end{center}

\vspace{30pt}


%% Hier wird der Titel der Arbeit eingetragen
\huge
\textsc{\theTitle}\\
\vspace{5pt}
\large
\textsc{\theSubTitle}\\
\vspace{40pt}

\normalsize
\textsc{\theCertificate}                                % BACHELORARBEIT 
\textsc{\\zur Erlangung des akademischen Grades\\}
\textsc{Bachelor of Engineering (B.Eng.)}\\

\vspace{40pt}


\textrm{\large \theAuthor}\\                            % Lukas Diedenhofen

\normalsize

\vspace{40pt}

Betreuerin:\\

\vspace{0pt}

\textrm{\large \theAdviserA}\\                          % Prof. Dr. rer. nat. Andrea Baumann

\vspace{30pt}

Tag der Abgabe: \thePublicationDate\\ 

\vspace{30pt}

eingereicht bei\\
Universität der Bundeswehr München\\
Fakultät für Elektrotechnik und Technische Informatik\\
\vspace{15pt}
\includegraphics[width=242pt]{images/logos/UniBwM_ETTI_6_geschnitten.pdf}


\vspace{25pt}
Neubiberg, \thePublicationMY 
\end{center}
\end{titlepage}
\textbf{}\cleardoublepage
%%%%%%%%%%%%%%%%%%%%%%%%%%%%%%%%%%%%%%%%%%%%%%%%%%%%%%%%%%%%%%%%%%%%%%%%%%%
%% Die Erklärung de Arbeit.                                              %%
%% In diesem Dokument sind keine Änderungen durch den Nutzer             %%
%% notwendig; Die Variablen werden aus der thesis.tex übernommen.        %%
%%%%%%%%%%%%%%%%%%%%%%%%%%%%%%%%%%%%%%%%%%%%%%%%%%%%%%%%%%%%%%%%%%%%%%%%%%%
\newpage
\chapter*{Erklärung}

\addcontentsline{toc}{chapter}{Erklärung}

\vspace{1cm}

Hiermit versichere ich, dass ich die vorliegende Arbeit selbstständig verfasst, noch nicht anderweitig für Prüfungszwecke vorgelegt und keine anderen als die angegebenen Quellen und Hilfsmittel benutzt habe, insbesondere keine anderen als die angegebenen Informationen. \\
\hspace{\fill}
\vspace{0.5cm}

Der Speicherung meiner Bachelorarbeit zum Zweck der Plagiatsprüfung stimme ich zu. Ich versichere, dass die elektronische Version mit der gedruckten Version inhaltlich übereinstimmt.\\
\hspace{\fill}
\vspace{0.6cm}

Neubiberg, den \thePublicationDate 
\hspace{\fill}
\newline\hspace{\fill}\newline\hspace{\fill}\newline\hspace{\fill}\newline\hspace{\fill}
\rule{0.3\textwidth}{0.4pt} \newline
\theAuthor
\hspace{\fill}
\cleardoublepage

%%%%%%%%%%%%%%%%%%%%%%%%%%%%%%%%%%%%%%%%%%%%%%%%%%%%%%%%%%%%%%%%%%%%%%%%%%%
%% Die folgenden Variable ist durch den Nutzer anzupassen.               %%
%% Sollte kein Abstract gewünscht sein, kann die folgende Variable       %%
%% auskommentiert werden.                                                %%
%%%%%%%%%%%%%%%%%%%%%%%%%%%%%%%%%%%%%%%%%%%%%%%%%%%%%%%%%%%%%%%%%%%%%%%%%%%
%%%%%%%%%%%%%%%%%%%%%%%%%%%%%%%%%%%%%%%%%%%%%%%%%%%%%%%%%%%%%%%%%%%%%%%
%%  Indoor-Navigation mit Smartphone-Sensorik                        %%
%%-------------------------------------------------------------------%%
%% Datei:        001_abstract.tex                                    %%
%% Beschreibung: Abstract für die Abschlussarbeit                    %%
%% Autor:        Maximilian Helfrich                                 %%
%% Datum:        01.05.2021                                          %%
%% Version:      1.0.0                                               %%
%%%%%%%%%%%%%%%%%%%%%%%%%%%%%%%%%%%%%%%%%%%%%%%%%%%%%%%%%%%%%%%%%%%%%%%

\begin{abstract}


    %% Anfang des Abstract-Texts
    Dieses Template soll als Formatvorlage für Bachelorarbeiten im Institut Etti6 dienen.
    Studierend können mithilfe dieser Vorlage, durch Ändern einiger weniger Variablen, eine konsistente Formatierung ihrer Abschlussarbeiten erreichen. Die zu verändernden Variablen sind im LaTeX-Quellcode entsprechend durch Kommentare markiert. Grundlegende LaTeX-Kenntnisse werden dennoch vorausgesetzt.
    %% Ende des Abstract-Texts
    
    
\end{abstract}

%%%%%%%%%%%%%%%%%%%%%%%%%%%%%%%%%%%%%%%%%%%%%%%%%%%%%%%%%%%%%%%%%%%%%%%%%%%
%%%%%%%%%%%%%%%%%%%%%%%%%%%%%%%%%%%%%%%%%%%%%%%%%%%%%%%%%%%%%%%%%%%%%%%%%%%

%% Inhaltsverzeichnis
\setcounter{tocdepth}{2}                %% Festlegen der Tiefe des Inhaltsverzeichnisses
\setcounter{secnumdepth}{2}

\tableofcontents                        %% Erzeugen des Inhaltsverzeichnisses
\cleardoublepage
\pagenumbering{arabic}

%%%%%%%%%%%%%%%%%%%%%%%%%%%%%%%%%%%%%%%%%%%%%%%%%%%%%%%%%%%%%%%%%%%%%%%%%%%
%% Die folgenden Variablen sind durch den Nutzer anzupassen.             %%
%% Hier werden die einzelnen Quelldokumente aus dem sourece-Folder       %%
%% referenziert, um im pdf-Dokument gerendert zu werden.                 %%
%%%%%%%%%%%%%%%%%%%%%%%%%%%%%%%%%%%%%%%%%%%%%%%%%%%%%%%%%%%%%%%%%%%%%%%%%%%
%% Hauptteil der Arbeit 
%%%%%%%%%%%%%%%%%%%%%%%%%%%%%%%%%%%%%%%%%%%%%%%%%%%%%%%%%%%%%%%%%%%%%%%
%%  Indoor-Navigation mit Smartphone-Sensorik                        %%
%%-------------------------------------------------------------------%%
%% Datei:        101_introduction.tex                                %%
%% Beschreibung: Titelseite für die Abschlussarbeit                  %%
%% Autor:        Maximilian Helfrich                                 %%
%% Datum:        01.05.2021                                          %%
%% Version:      1.0.0                                               %%
%%%%%%%%%%%%%%%%%%%%%%%%%%%%%%%%%%%%%%%%%%%%%%%%%%%%%%%%%%%%%%%%%%%%%%%
\chapter{Anleitung} \index{Anleitung}

Diese Anleitung bietet einen Überblick über das Template für Bachelorarbeiten im Institut etti 6, und ist selbst ein Beispiel für die Anwendung Dieser. 

\section{Titelblatt und Erklärung}\index{Titelblatt und Erklärung}\label{Titelblatt und Erklärung}

Das Titelblatt und die Erklärung sind bereits erstellt, und die Quelldateien müssen nicht mehr angepasst werden. Die Belegung der Variablen für Name, Datum, etc. findet im Einstiegspunkt, der Datei thesis.tex statt. Die zu ändernden Variablen sind durch Kommentare markiert.

\section{Abstract}\index{Abstract}\label{Abstract}

Standardmäßig wird in diesem Template ein Abstract erzeugt. Sollte dies nicht gewünscht sein, kann die Erzeugung des Abstract in der Datei thesis.tex durch auskommentieren deaktiviert werden. Der Text des Abstract kann in der Datei source\slash002\textunderscore abstract.tex angepasst werden.

\section{Inhaltsverzeichnis}\index{Inhaltsverzeichnis}\label{Inhaltsverzeichnis}

Das Inhaltsverzeichnis wird automatisch erzeugt. Die Tiefe kann in der Datei thesis.tex angepasst werden.

\section{Hauptteil}\index{Hauptteil}\label{Hauptteil}

Es bietet sich an, für jedes Kapitel eine eigene .tex-Datei im source-Ordner zu erzeugen. Um diese einzubinden, müssen sie in der Datei thesis.tex referenziert werden.

\begin{figure}[H]
    \centering
    \includegraphics[width={\textwidth/2}]{images/pictures/LaTeX_logo.svg.png}
    \caption{Beispielabbildung}
    \label{figure:example}
\end{figure}

\section{Anhang}\index{Anhang}\label{Anhang}

\subsection{Abbildungen}\index{Abbildungen}\label{Abbildungen}

Das Abbildungsverzeichnis wird automatisch erstellt, als Beispiel hierfür dient die oben gezeigte Beispielabbildung. Dies kann verhindert werden, indem die entsprechende Anweisung in der Datei 801\textunderscore appendix.tex auskommentiert wird.

\subsection{Abkürzungen und Glossar}\index{Abkürzungen und Glossar}\label{Abkürzungen und Glossar}

Abkürzungen wie etwa \acrshort{ba} und Erläuterungen wie \Gls{latex}-Template können in der Datei acronym.tex ergänzt werden. Die entsprechenden Verzeichnisse im Anhang werden automatisch erzeugt.


\subsection{Quellcode}\index{Quellcode}\label{Quellcode}
{\setstretch{1}
\lstinputlisting[
    style=Java, 
    language=c, 
    caption={thesis.js}
    \label{lst:jsThesis}
]{code/thesis.js}
}\index{thesis.js}

\subsection{Literaturverzeichnis}\index{Literaturverzeichnis}\label{Literaturverzeichnis}
Das Literaturverzeichnis wird automatisch erstellt. Dafür wird eine Literaturdatenbank (Dateiformat .bib) benötigt. Empfehlenswert hierfür sind die Programme Zitavi (Windows) oder Zotero (MacOs / Linux). Aus diesen können diese erzeugten Bibliotheken exportiert, und in die oberste Ebene der Ordnerstruktur des Templates unter dem Namen Literatur.bib eingefügt werden. Zitate, die im Text vorkommen, werden automatisch in das Literaturverzeichnis übernommen.
Kennzeichnen von indirekten Zitaten \cite{noauthor_software_nodate}, und "direkten Zitaten" \cite[s.~234]{noauthor_software_nodate} erzeugt somit dem IEEE-Zitierstil entsprechenden Zitate.
\newpage

\subsection{Tabellenverzeichnis}\index{Tabellenverzeichnis}\label{Tabellenverzeichnis}
Das Tabellenverzeichnis wird automatosch erstellt, wie die Folgende Beispieltablle zeigt:

\begin{table}[h!]
\centering
 \begin{tabular}{||c c c c||} 
 \hline
 a & b & c & d \\ [0.5ex] 
 \hline\hline
 1 & 6 & 11 & 16 \\ 
 2 & 7 & 12 & 17 \\
 3 & 8 & 13 & 18 \\
 4 & 9 & 14 & 19 \\
 5 & 10 & 15 & 20 \\ [1ex] 
 \hline
 \end{tabular}
 \caption{Beispieltabelle}
\end{table}

\chapter{Einleitung}\index{Einleitung}
In der modernen Zeit der Digitalisierung spielt die effiziente und effektive Verarbeitung von Daten eine entscheidende Rolle. Dabei stellt sich auch die Frage, wie Datensätze am besten dargestellt werden können, um die richtigen Informationen leicht erkennbar zu machen. Neben den bekannten Formen wie Balken-, Kreis- oder Liniendiagrammen besteht in der Informatik auch die Möglichkeit, Datensätze in Form eines Graphen zu präsentieren.


\section{Hintergrund und Motivation}\index{Hintergrund und Motivation}\label{Hintergrund und Motivation}
In einer Projektarbeit, welcher dieser Bachelorarbeit vorangegangen ist, habe ich mich mit der Aufgabe beschäftigt kleine Gruppen von Profifussballspielern in  einem kompletten Graph zu entdecken, welche häufig zusammen den Verein wechseln. Bei dieser Aufgabe wurden qualitativ leider keine guten Ergebnisse generiert, worauf hin sich folgende Fragestellung  entwickelte, mit der ich mich in dieser wissenschaftlichen Arbeit beschäftigen will. Wie können Community Detection Algorithmen effektiv und effizient angewendet werden um optimale Ergebnisse zu erzielen, oder gibt es eine Möglichkeit die gefunden Gruppen in Netzwerken mit einer Eavulation zu Bewerten um eine Aussage über die Qualität und Einteilung der Gruppen zu machen.

\section{Ziele}\index{Ziele}\label{Ziele}
Das Hauptziel dieser Bachelorarbeit ist es, eine umfassende Bewertung von Community Detection Algorithmen mithilfe der CDlib-Bibliothek durchzuführen. Dabei liegt der Fokus auf der Analyse und Vergleich der Leistungsfähigkeit verschiedener Algorithmen bei der Erkennung von Gemeinschaftsstrukturen in komplexen Netzwerken. Die Arbeit strebt an, Erkenntnisse über die Stärken und Schwächen der Algorithmen zu gewinnen und ihre Anwendbarkeit in verschiedenen Domänen zu untersuchen. Darüber hinaus sollen konkrete Anwendungsbeispiele der CDlib-Bibliothek vorgestellt werden, um ihre Wirksamkeit in realen Szenarien zu demonstrieren. Das langfristige Ziel ist es, zur Weiterentwicklung und Verbesserung von Community Detection Techniken beizutragen und eine solide Grundlage für zukünftige Forschung und Anwendungen auf diesem Gebiet zu schaffen.\\
\\
Mit dieser wissenschaftlichen Arbeit verfolge ich das Hauptziel eine qualitative und möglichst umfangreiche Bewertung von Community Detection Algorithmen durchzuführen. Für Arbeit der Evaluation stütze ich mich auf die Funktionalitäten der CDlib-Bibliothek. Die Bewertung soll sich dabei vorallem auf die Leistungsfähigkeit der Techniken zur Gruppenerkennung stützen. Um die Ergebnisse gut miteinander vergleichen zu können werden die Verschiedenen Algorithmen an unterschiedlichen und künstlich als auch realen Datensätzen getested. Durch diese Bewertungen soll es möglich sein die Einsatzgebiete der Algorithmen noch stärker einzugrenzen.

\chapter{Grundlagen}\index{Grundlagen}
Im weiteren Verlauf dieser schriftlichen Arbeit, versuche tiefergehende Themen und Strukturen der Netzwerk- bzw. Graphentheorie schrittweise zu erklären. Allerdings setze ich Grundkenntnisse der Informatik und ein gutes Verständnis der Datenstrukturen von Graphen voraus. 

Die Analyse von Netzwerken entwickelt sehr schnell zu einer komplexen und unter Umständen zeitaufwendigen Aufgabe. Aufgrund der rasant ansteigenden Menge an Daten die dazu führen, das Netzwerke immer umfangreicher werden ist es nicht nur ein Zeitproblem. Mit der heutigen Zeit und der Möglichkeit auch immer mehr Details in Netzwerkgraphen mit zu berücksichtigen gibt es eine vielzahl an Techniken die angewendet werden können.

\section{Community Detection}\index{Community Detection}\label{Community Detection}
Die Erkennung von Gruppen in Netzwerken wird dazu verwendet Graphen zu analysieren oder zusammenhängende Information zu extrahieren. Je nach Zielsetzung kann es auch als eine Art der Klassifiezierung verwendet werden. Doch wie bei vielen anderen Themen gibt es auch bei der Community Detection ein Problem das bewältigt werden muss. Es kann nicht einfach eine deterministische Vorgehensweise angewendet werden, denn es gibt keine einheitliche Definition wie Gruppen erkannt werden können.  Für Communityies dagegen gibt es zumindest klarere Regeln wie diese definiert werden können. So zeichnet sich eine Gruppe durch die kompakte und gegenseitige Bindung, sowie eine klare Abgrenzung nach außen aus. Hierfür lassen sich einige Methriken gut verwenden, welche im entsprechenden Nachfolgendem Kapitel näher erläutert werden

//TODO Definition Community, Verweis auf Methriken.

\subsection{Definition}\index{Definition}\label{Definition}
DELETE ?? Inhalte können in den Absatz "Community Detection" geschrieben werden.

\subsection{Methriken}\index{Methriken}\label{Methriken}

\subsubsection{Modularity}\index{Modularity}\label{Modularity}
//TODO read Fortunato 2010

\subsubsection{Density}\index{Density}\label{Density}
//TODO read Zhenping Li et al 2008
Die Dichte (engl. density) ist ein Maß, womit angegeben wird wie stark die Verzweigung in einem Graph oder Teilgraph ist. Hiermit lassen sich schon relativ einfach und schnell kleine Gruppen definieren. Berechnet wir sie im allgemeinen wie folgt. \[ d = {2m \over n(n-1)} \] Berechnung der Dichte des Graphen mit m Kanten und n Knoten. 

Liegen schon Informationen vor, aus welchen Kanten und Konten eine Community besteht kann natürlich auch hiervon die Dichte dieser bestimmt werden. \[ d_{com} = {2m_{com} \over n_{com}(n_{com}-1)} \] Auch hier entspricht m den Kanten und n den Knoten.

Darüber hinaus kann auch die Dichte berechnet werden, wie stark eine Gruppe nach außen zum restlichen Graphen verbunden ist. \[ d_{ext} = {2m_{ext} \over n_{com}(n-n_{com})} \]

\subsubsection{Edge Betweeness}\index{Edge Betweeness}\label{Edge Betweeness}
Edge Betweeness ist ein Maß dafür wie wichtig eine Kante ist. Es wird die Anzahl der Durchläufe gemessen.

Edge Betweenness gibt das Verhältnis an, wie oft eine Kante auf einem Kürzesten Weg  zwischen zwei Knoten durchlaufen wird. \[E_B(e)= \sum_{s \neq t \in V}\frac{\sigma_{st}(e)}{\sigma_{st}}\] Es gilt somit als verlässliche Ausage darüber wie wichtig eine Kante im Graphen ist. Je höher der Wert desto öfter wir die Kante auf der Suche nach dem kürzesten Pfad durch einen Graphen passiert.


\subsubsection{Degree Centrality}\index{Degree Centrality}\label{Degree Centrality}
Die Degree Centrality gibt an wie gut ein einzelner Konten innerhalb des Graphes oder seiner Community verzweigt ist. \[c(n) = {deg(n) \over N-1}\]

\subsection{Algorithmen}\index{Algorithmen}\label{Algorithmen}

Liste der Algorithmen:
Girvan-Newman
Fast Greedy Modularity
Markov Cluster
Random Walk
Louvain
Label Propagation
k-Clique

\section{NetworkX}\index{NetworkX}\label{NetworkX}
NetworkX ist eine Bibliothek für die Scriptsprache Python und bietet umfassende Möglichkeiten Datensätze in Form von Graphen zu visuallisieren. Es werden darüber hinaus auch Funktionen angeboten um künstliche bzw. zufällige Netzwerke zu generieren.

\section{CDlib - Biblitohek}\index{CDlib - Bibliothek}\label{CDlib - Bibliothek}
Die CDlib Bibliothek wird verwendet für die Bearbeitung der Hauptaufgabe. Die Bewertung der Algorithmen. Auch wird sie dazu herangezogen, verschiedene künstliche Datensätze für Graphen zu generieren. Hierbei stüzt sie sich aber hauptsächlich auf bereits fertig implementierte Funktionen aus der vorherig genannten Bibliothek NetworkX. Dennoch unterscheiden sich die grundsätzlichen Funktionen und der Umfang dieser beiden Bibliotheken stark voneinander. Zum Vergleich, sind in CDlib jedoch mehr Algorithmen zur Erkennung von Netzwerken implementiert und darüber hinaus sind auch die einzelnen Bewertungsfunktionen, die für Analyse der Ergebnisse essentiell sind.

\chapter{Durchführung der Bewertung}\index{Durchführung der Bewertung}

\section{Zusammenstellung der Testdaten}\index{Zusammenstellung der Testdaten}\label{Zusammenstellung der Testdaten}

\subsection{Synthetische Daten}\index{Synthetische Daten}\label{Synthetische Daten}

\subsection{Reale Daten}\index{Reale Daten}\label{Reale Daten}

\section{Bewertungsfunktionen}\index{Bewertungsfunktionen}\label{Bewertungsfunktionen}

\section{Experimente}\index{Experimente}\label{Experimente}

\chapter{Analyse und Bewertung}\index{Analyse und Bewertung}

\section{Ergebnisse der Algorithemen und Bewertungsfunktionen}\index{Ergebnisse der Algorithemen und Bewertungsfunktionen}\label{Ergebnisse der Algorithemen und Bewertungsfunktionen}

\section{Vergleich der Ergebnisse}\index{Vergleich der Ergebnisse}\label{Vergleich der Ergebnisse}

\chapter{Zusammenfassung}\index{Zusammenfassung}

\section{Diskussion der Ergebnisse}\index{Diskussion der Ergebnisse}\label{Diskussion der Ergebnisse}

\section{Grenzen der Studie}\index{Grenzen der Studie}\label{Grenzen der Studie}

\section{Ausblick}\index{Ausblick}\label{Ausblick}

\cleardoublepage 
\pagenumbering{Roman}
\cleardoublepage

% Erstellen des Anhangs

\chapter*{Anhang}

\cleardoublepage

%% Abkürzungsverzeichnis

\printglossary[type=\acronymtype]
\printglossary

\listoffigures
\cleardoublepage 

%% Tabellenverzeichnis
\listoftables
\cleardoublepage 

%% Quellcodeverzeichnis
\renewcommand\lstlistlistingname{Quellcodeverzeichnis} 
\lstlistoflistings 
\renewcommand*\lstlistingname{Quellcode}
\cleardoublepage 

%% Stichwortverzeichnis, Index
\renewcommand{\indexname}{Stichwortverzeichnis}
\printindex
\cleardoublepage 

\printbibliography
\cleardoublepage 

\end{document}